\documentclass[10pt, a4paper, twoside]{book}
\usepackage[margin=2.5cm]{geometry}
\usepackage{multicol, enumitem}
\usepackage[most]{tcolorbox}
\usepackage{dialogue}
\usepackage{gfsartemisia-euler}
\usepackage[T1]{fontenc}

\newtcolorbox{mytextbox}[1][]{%
	sharp corners,
	enhanced,
	colback=white,
	attach title to upper,
	#1
}

\setcounter{chapter}{-1}

\title{\textbf{D6 System}}
\author{West End Games \\ transcribed by Kosmonaut}
\begin{document}
	\maketitle
	\tableofcontents
	\chapter{INTRODUCTION}
	\begin{multicols}{2}
		For those of you who have already played roleplaying games you may want to skip ahead to Chapter One, "The Player's Section" For the newcomers. The remainder of this introductory chapter explains the concepts of roleplaying, characters and adventures. Get your imagination fired up -- you're going to need it!
		\section*{SO, WHAT \emph{IS} ROLEPLAYING?}
		You've read novels; you've seen movies; you've watched television. What is it about these things that so compels people to sit down time and again to spend some of our precious free moments? \par Stories. \par We love to watch characters fall into conflict, and we cannot wait to see how things turn out. Does the inept but lovable protagonist get the girl? Does the secret agent make it out of the villain's lair before the whole place explodes? Does the noble hero sacrifice himself to save his kingdom? \par The genre doesn't matter. Believable, conflict-oriented stories engage our senses so deeply that we find ourselves hypnotized by the events unfolding before us. 
		\begin{mytextbox}
			\subsection*{THE GAME IN A NUTSHELL}
			Roll some six-sided dice (the number of which is determined by your character's statistics), add them up, and compare the total to the difficulty for the task you want the character to accomplish. If the roll is equal or higher than the difficulty, the character succeeds. If it's lower, he fails. That's it!
		\end{mytextbox}	
		In this day of virtual reality and interactive television, we can now become involved in these stories that so grip us. We become characters in tales of our own design. stories propelled forward by our actions and reactions and moving ever closer to the inevitable conclusion.\par Think of roleplaying as interactive storytelling. You and a group of friends create alter ego and enter the world of your imaginations, exploring realms limited only by your creativity. One person assumes the role of referee -- or gamemaster - informing the players of their characters' situations -- the environment, the words and actions of the other characters present (those characters not controlled by the players), and the results of the players' characters' activities. \par 
		With a roleplaying game you can thrust yourself into the heart of adventure, becoming characters in worlds of your own imagination or your favourite novel, movie, or television show, like \textit{Star Wars}, \textit{Dr. Who}, \textit{James Bond}, \textit{Babylon 5}, \textit{Space: Above and Beyond}, \textit{Hercules: The Legendary Journeys}, or \textit{The X-Files}, to name just a few.\footnote{All copyrights and trademarks are the property of their respective owners.} \par Let's stop a moment and look in on a typical game session. Judy, Mike, Chris, and Tim have gotten together on a Friday evening to continue their fantasy \textit{campaign} (a linked series of adventures -- like individual books in a novel series). Tim, the gamemaster (GM), has just begun to summarize the events of the last adventure (for now don't worry about the game mechanics -- the dice rolling, character attribute and skill names, et cetera). The dialogue in quotes indicates that the player is speaking in the voice of his character. 
		\begin{dialogue}
			\speak{Tim} Your characters escaped the Vizier's Flying Armada, but your skyship suffered sever damage during the engagement. You're going to have to make repairs soon.
			\speak{Judy \textit{(playing Captain Walker)}} I walk to the prow and use my spyglass to look for a good place to anchor the ship.
			\speak{Tim} About a quarter of a mile north is a large plain dotted with several smallish buildings. You can't make out much more than  that from this distance.
			\speak{Mike \textit{(playing First Mate Stensson)}} "Captain, what're yer orders?"
			\speak{Judy} "Head for that clearing" (\textit{points})
			\speak{Mike} "Aye Captain."
			\speak{Tim} Chris, where's your character at this point?
			\speak{Chris \textit{(playing Crewman Fahrer)}} In the crow's nest.
			\speak{Tim} Okay, make a Perception check.
			\speak{Chris} (\textit{Rolls some dice.}) I rolled a 14.
			\speak{Tim} A fourteen? Okay, that's pretty good. You notice a plume of grayish-white smoke twisting upward into the sky from a wooded hill to the northeast.
			\speak{Chris} "Captain, look: a fire to the northeast."
			\speak{Judy} Do I see the smoke?
			\speak{Tim} Yes, it's about a half-mile from your current position.
			\speak{Judy} "Stensson, what do you think? Should we check out the fire first or head to the village?"
			\speak{Chris} Do I think the ship can make it to the hill and then back to the village?
			\speak{Tim} Make a \textit{shipwright} roll.
			\speak{Chris} (\textit{Rolls some dice.}) Uh-oh. I only rolled a three.
			\speak{Tim} Well it's hard for you to tell. A lot of the damage is on the lower hull, which you can't see very well from the deck. Mike, make a Perception roll.
			\speak{Mike} Okay. (\textit{Rolls.}) I got a 9.
			\speak{Tim} That's good. Something tells you to look behind you. When you turn you see a glint of light in the sky. It only takes you a couple of seconds to realize it's the Armada's lead ship.
			\speak{Mike} "Look, it's tje Armada."
			\speak{Chris} I'm climbing down from the crow's nest.
			\speak{Judy} "Get to your battle stations while I turn us around. we can't outrun them, so we're going to have to fight."
			\speak{Tim} All right, we're getting into combat rounds now. Everyone make Reflexes rolls...
		\end{dialogue}
		The night continues with an exciting confrontation between the Armada and the player's characters. Luckily, the pirate allies of the characters show up to help out -- of course, they wait until the very las moment to arrive!
		\section*{GETTING READY TO PLAY}
		Think of roleplaying as a combination of interactive storytelling, acting improvisation, and dice-rolling. You and your friends are writing your own stories, filling them with exotic locales, interesting characters, and evocative scenes. All you need is some paper, a pencil, dice, and your imagination!
	\end{multicols}
	\chapter{THE PLAYER'S SECTION}
	\begin{multicols}{2}
		Participating in a roleplaying game takes a slight amount of knowledge about how the game works. Most of the time you can rely on the gamemaster to coach you through, but it makes life a lot easier if all the players know at least the basics. \par The details covered here appear in the rest of the book (Players, you don't need to worry about those chapters; only the gamemaster has to know the fine points.) Once you've read this section, you'll be armed and ready to become part of a roleplaying adventure. The D6 System is designed so that you can play in any genre (science fiction, fantasy, cyberpunk, Victorian, pulp, horror, et cetera) without having to learn a new set of rules for each one! \par Now let's begin your foray into the exciting world of roleplaying games...
		\section*{CHARACTERS}
		To play the game you'll need a character. A character is an alter ego whose part you will assume for the duration of the gaming session. Think of it as improvisational acting: you know the abilities and personality of your character and decide how that character reacts to the situation presented to him. Unlike most traditional games, which follow set procedures for each player's turn, roleplaying games leave all options open. If someone shoots at your character, for example, you can decide to leap out of the way, or return fire, or catch the bullet in your teeth... \par Okay, that last option sounds pretty outlandish, but what if your character is a superhero? And then again, what if he isn't? \par So, we need a way to quantify the character's abilities -- his aptitudes, skills, special powers (magical, psychic, super), et cetera. The D6 System represents your character's level of ability in each area with a \textit{die code}, a number of six-sided dice plus a number of "pips". For example, a die code of 3D+1 means three six-sided dice plus one pip (don't worry about what you do with these die codes for now; we'll cover that a little later in the section titled \textit{Making Dice Rolls}). All you need to know right now is that the more dice and the more pips, the better the character's expertise in the particular aptitude or skill. \par 
		\begin{mytextbox}
			\subsection*{CHREATING A CHARACTER}
			{\small 
				\begin{enumerate}[wide, labelwidth=!, labelindent=0pt]
					\setlength\itemsep{-0.4em}
					\item Make a photocopy of a character sheet.
					\item Distribute attribute dice.
					\item Select skills and distribute skill dice.
					\item Roll for body points (if applicable).
					\item Determine personal information (name, species, gender, height, weight, appearance)
					\item Choose Advantages and Disadvantages.
					\item Record or select special abilities (spells, psychic powers, et cetera).
					\item Create background and personality.
					\item Determine starting money and purchase equipment.
				\end{enumerate}
			}
		\end{mytextbox}
		\subsection*{ATTRIBUTES}
		Attributes represent a character's basic aptitudes -- her inherent levels of ability in various areas, from physical strength to logical reasoning. Your gamemaster will provide you with either a character template (a partially created character that you can customize and use as your own) or a list of attributes that will be used for this game world (so you can create a character from scratch).
		\textit{\begin{enumerate}
			\setlength\itemsep{-0.4em}
			\item[] \textbf{Example: Space Opera Game Attributes}
			\item[] Strength: overall strength and level of physical conditioning
			\item[] Reflexes: reaction time
			\item[] Coordination: aim and balance
			\item[] Perception: observation and sixth-sense
			\item[] Reasoning: deduction and problem-solving
			\item[] Knowledge: education (formal and informal)
		\end{enumerate}}
		Characters begin with a total number of dice dictated by the gamemaster, usually three dice per attribute. In our example, a starting character would have a total of eighteen dice (18D). You decide how those dice should be divided among the character's attributes. If you want to create a space smuggler, for example, you'll probably concentrate your available dice on the character's Strength, Reflexes and Perception Attributes, the aptitudes most important to someone with that career.
		\textit{\begin{enumerate}
				\setlength\itemsep{-0.4em}
				\item[] \textbf{Example: Space Smuggler Character -- 18 Total Dice}
				\item[] Strength: 4D
				\item[] Reflexes: 4D
				\item[] Coordination: 2D
				\item[] Perception: 4D
				\item[] Reasoning: 2D
				\item[] Knowledge: 2D
		\end{enumerate}}
		You might have noticed that none of these die codes have pips. Well you \emph{can} break up these dice into smaller units (just like you can break tens into ten ones). Each die code has three levels of pips: 0, 1, and 2. The progression looks like this: 0, +1, +2, 1D+0, 1D+1, 1D+2, 2D+0, 2D+1, 2D+2, 3D+0, 3D+1, 3D+2, 4D+0 et cetera.
	\end{multicols}
\end{document}