\documentclass[10pt, a4paper, twoside]{book}
\usepackage[margin=2.5cm]{geometry}
\usepackage{multicol, enumitem}
\usepackage[most]{tcolorbox}
\usepackage{dialogue}
\usepackage{gfsartemisia-euler}
\usepackage[T1]{fontenc}

\newtcolorbox{mytextbox}[1][]{%
	sharp corners,
	enhanced,
	colback=white,
	attach title to upper,
	#1
}

\setcounter{chapter}{-1}

\title{\textbf{D6 System}}
\author{West End Games \\ transcribed by Kosmonaut}
\begin{document}
	\begin{titlepage}
		\pagestyle{empty}
		\begin{center}
			\textbf{\huge{The D6 System}}
		\end{center}
	\end{titlepage}
	\begin{center}
		{\large 
			\textsc{
				Design \& Development \\ \textbf{ George Strayton} \\
				\vspace{1cm}
				Editing \\ \textbf{Greg Farshtey} \\
				\vspace{1cm}
				Playtesting \& Advice \\ \textbf{Mark Barnabo, Chris Doyle, Peter Schweighofer, Paul Sudlow, Eric S. Trautmann} \\
				\vspace{1cm}
				\hrule
				\vspace{1cm}
				Transcription \\ \textbf{KosmonautFX}
			}
		}
	\end{center}
	\pagestyle{plain}
	\tableofcontents
	
	\frontmatter
	
	\mainmatter
	\chapter{INTRODUCTION}
	\begin{multicols}{2}
		For those of you who have already played roleplaying games you may want to skip ahead to Chapter One, "The Player's Section" For the newcomers. The remainder of this introductory chapter explains the concepts of roleplaying, characters and adventures. Get your imagination fired up -- you're going to need it!
		\section*{SO, WHAT \emph{IS} ROLEPLAYING?}
		You've read novels; you've seen movies; you've watched television. What is it about these things that so compels people to sit down time and again to spend some of our precious free moments? \par Stories. \par We love to watch characters fall into conflict, and we cannot wait to see how things turn out. Does the inept but lovable protagonist get the girl? Does the secret agent make it out of the villain's lair before the whole place explodes? Does the noble hero sacrifice himself to save his kingdom? \par The genre doesn't matter. Believable, conflict-oriented stories engage our senses so deeply that we find ourselves hypnotized by the events unfolding before us. 
		\begin{mytextbox}
			\subsection*{THE GAME IN A NUTSHELL}
			Roll some six-sided dice (the number of which is determined by your character's statistics), add them up, and compare the total to the difficulty for the task you want the character to accomplish. If the roll is equal or higher than the difficulty, the character succeeds. If it's lower, he fails. That's it!
		\end{mytextbox}	
		In this day of virtual reality and interactive television, we can now become involved in these stories that so grip us. We become characters in tales of our own design. stories propelled forward by our actions and reactions and moving ever closer to the inevitable conclusion.\par Think of roleplaying as interactive storytelling. You and a group of friends create alter ego and enter the world of your imaginations, exploring realms limited only by your creativity. One person assumes the role of referee -- or gamemaster - informing the players of their characters' situations -- the environment, the words and actions of the other characters present (those characters not controlled by the players), and the results of the players' characters' activities. \par 
		With a roleplaying game you can thrust yourself into the heart of adventure, becoming characters in worlds of your own imagination or your favourite novel, movie, or television show, like \textit{Star Wars}, \textit{Dr. Who}, \textit{James Bond}, \textit{Babylon 5}, \textit{Space: Above and Beyond}, \textit{Hercules: The Legendary Journeys}, or \textit{The X-Files}, to name just a few.\footnote{All copyrights and trademarks are the property of their respective owners.} \par Let's stop a moment and look in on a typical game session. Judy, Mike, Chris, and Tim have gotten together on a Friday evening to continue their fantasy \textit{campaign} (a linked series of adventures -- like individual books in a novel series). Tim, the gamemaster (GM), has just begun to summarize the events of the last adventure (for now don't worry about the game mechanics -- the dice rolling, character attribute and skill names, et cetera). The dialogue in quotes indicates that the player is speaking in the voice of his character. 
		\begin{dialogue}
			\speak{Tim} Your characters escaped the Vizier's Flying Armada, but your skyship suffered sever damage during the engagement. You're going to have to make repairs soon.
			\speak{Judy \textit{(playing Captain Walker)}} I walk to the prow and use my spyglass to look for a good place to anchor the ship.
			\speak{Tim} About a quarter of a mile north is a large plain dotted with several smallish buildings. You can't make out much more than  that from this distance.
			\speak{Mike \textit{(playing First Mate Stensson)}} "Captain, what're yer orders?"
			\speak{Judy} "Head for that clearing" (\textit{points})
			\speak{Mike} "Aye Captain."
			\speak{Tim} Chris, where's your character at this point?
			\speak{Chris \textit{(playing Crewman Fahrer)}} In the crow's nest.
			\speak{Tim} Okay, make a Perception check.
			\speak{Chris} (\textit{Rolls some dice.}) I rolled a 14.
			\speak{Tim} A fourteen? Okay, that's pretty good. You notice a plume of grayish-white smoke twisting upward into the sky from a wooded hill to the northeast.
			\speak{Chris} "Captain, look: a fire to the northeast."
			\speak{Judy} Do I see the smoke?
			\speak{Tim} Yes, it's about a half-mile from your current position.
			\speak{Judy} "Stensson, what do you think? Should we check out the fire first or head to the village?"
			\speak{Chris} Do I think the ship can make it to the hill and then back to the village?
			\speak{Tim} Make a \textit{shipwright} roll.
			\speak{Chris} (\textit{Rolls some dice.}) Uh-oh. I only rolled a three.
			\speak{Tim} Well it's hard for you to tell. A lot of the damage is on the lower hull, which you can't see very well from the deck. Mike, make a Perception roll.
			\speak{Mike} Okay. (\textit{Rolls.}) I got a 9.
			\speak{Tim} That's good. Something tells you to look behind you. When you turn you see a glint of light in the sky. It only takes you a couple of seconds to realize it's the Armada's lead ship.
			\speak{Mike} "Look, it's tje Armada."
			\speak{Chris} I'm climbing down from the crow's nest.
			\speak{Judy} "Get to your battle stations while I turn us around. we can't outrun them, so we're going to have to fight."
			\speak{Tim} All right, we're getting into combat rounds now. Everyone make Reflexes rolls...
		\end{dialogue}
		The night continues with an exciting confrontation between the Armada and the player's characters. Luckily, the pirate allies of the characters show up to help out -- of course, they wait until the very las moment to arrive!
		\section*{GETTING READY TO PLAY}
		Think of roleplaying as a combination of interactive storytelling, acting improvisation, and dice-rolling. You and your friends are writing your own stories, filling them with exotic locales, interesting characters, and evocative scenes. All you need is some paper, a pencil, dice, and your imagination!
	\end{multicols}
	\chapter{THE PLAYER'S SECTION}
	\begin{multicols}{2}
		Participating in a roleplaying game takes a slight amount of knowledge about how the game works. Most of the time you can rely on the gamemaster to coach you through, but it makes life a lot easier if all the players know at least the basics. \par The details covered here appear in the rest of the book (Players, you don't need to worry about those chapters; only the gamemaster has to know the fine points.) Once you've read this section, you'll be armed and ready to become part of a roleplaying adventure. The D6 System is designed so that you can play in any genre (science fiction, fantasy, cyberpunk, Victorian, pulp, horror, et cetera) without having to learn a new set of rules for each one! \par Now let's begin your foray into the exciting world of roleplaying games...
		\section*{CHARACTERS}
		To play the game you'll need a character. A character is an alter ego whose part you will assume for the duration of the gaming session. Think of it as improvisational acting: you know the abilities and personality of your character and decide how that character reacts to the situation presented to him. Unlike most traditional games, which follow set procedures for each player's turn, roleplaying games leave all options open. If someone shoots at your character, for example, you can decide to leap out of the way, or return fire, or catch the bullet in your teeth... \par Okay, that last option sounds pretty outlandish, but what if your character is a superhero? And then again, what if he isn't? \par So, we need a way to quantify the character's abilities -- his aptitudes, skills, special powers (magical, psychic, super), et cetera. The D6 System represents your character's level of ability in each area with a \textit{die code}, a number of six-sided dice plus a number of "pips". For example, a die code of 3D+1 means three six-sided dice plus one pip (don't worry about what you do with these die codes for now; we'll cover that a little later in the section titled \textit{Making Dice Rolls}). All you need to know right now is that the more dice and the more pips, the better the character's expertise in the particular aptitude or skill. \par 
		\begin{mytextbox}
			\subsection*{CHREATING A CHARACTER}
			{\small 
				\begin{enumerate}[wide, labelwidth=!, labelindent=0pt]
					\setlength\itemsep{-0.4em}
					\item Make a photocopy of a character sheet.
					\item Distribute attribute dice.
					\item Select skills and distribute skill dice.
					\item Roll for body points (if applicable).
					\item Determine personal information (name, species, gender, height, weight, appearance)
					\item Choose Advantages and Disadvantages.
					\item Record or select special abilities (spells, psychic powers, et cetera).
					\item Create background and personality.
					\item Determine starting money and purchase equipment.
				\end{enumerate}
			}
		\end{mytextbox}
		\subsection*{ATTRIBUTES}
		Attributes represent a character's basic aptitudes -- her inherent levels of ability in various areas, from physical strength to logical reasoning. Your gamemaster will provide you with either a character template (a partially created character that you can customize and use as your own) or a list of attributes that will be used for this game world (so you can create a character from scratch).
		\textit{\begin{enumerate}
			\setlength\itemsep{-0.4em}
			\item[] \textbf{Example: Space Opera Game Attributes}
			\item[] Strength: overall strength and level of physical conditioning
			\item[] Reflexes: reaction time
			\item[] Coordination: aim and balance
			\item[] Perception: observation and sixth-sense
			\item[] Reasoning: deduction and problem-solving
			\item[] Knowledge: education (formal and informal)
		\end{enumerate}}
		Characters begin with a total number of dice dictated by the gamemaster, usually three dice per attribute. In our example, a starting character would have a total of eighteen dice (18D). You decide how those dice should be divided among the character's attributes. If you want to create a space smuggler, for example, you'll probably concentrate your available dice on the character's Strength, Reflexes and Perception Attributes, the aptitudes most important to someone with that career.
		\textit{\begin{enumerate}
				\setlength\itemsep{-0.4em}
				\item[] \textbf{Example: Space Smuggler Character -- 18 Total Dice}
				\item[] Strength: 4D
				\item[] Reflexes: 4D
				\item[] Coordination: 2D
				\item[] Perception: 4D
				\item[] Reasoning: 2D
				\item[] Knowledge: 2D
		\end{enumerate}}
		You might have noticed that none of these die codes have pips. Well you \emph{can} break up these dice into smaller units (just like you can break tens into ten ones). Each die code has three levels of pips: 0, 1, and 2. The progression looks like this: 0, +1, +2, 1D+0, 1D+1, 1D+2, 2D+0, 2D+1, 2D+2, 3D+0, 3D+1, 3D+2, 4D+0 et cetera. Since any number plus zero equals that number, we can drop the +0 pips, leaving us with: +1, +2, 1D, 1D+1, 1D+2, 2D, 2D+1 et cetera. We can then divide one die into three +1's, or a +1 and a +2. Just remember that three pips equals one die(1D=+3).(Don't worry, it's not as complicated as it seems.)
		\textit{\begin{enumerate}
				\setlength\itemsep{-0.4em}
				\item[] \textbf{Example: Revised Space Smuggler Character -- 18 Total Dice}
				\item[] Strength: 4D
				\item[] Reflexes: 3D+2
				\item[] Coordination: 2D+1
				\item[] Perception: 4D
				\item[] Reasoning: 2D
				\item[] Knowledge: 2D
		\end{enumerate}}
		Let's check our math. First we'll add the dice (4D+3D+2D+4D+2D+2D=17D) and then the pips (2+1=3=1D) for a total of 18D (17D+1D=18D). \par Attributes typically have a lower limit of 2D and an upper limit of 4D, with 3D the average. Special circumstances can change those boundaries -- ask your gamemaster about them if you're interested (or read the \textit{Characters} chapter of the Gamemaster Section). \par Still with us? Good. Don't worry, the die code progression is the most difficult part of the game. Once you've got that, everything else is simple.
		\subsection*{SKILLS}
		At this point you've quantified the character's basic aptitudes. But what about the specific areas he has either studied, practised, or been trained in? We need some way to represent these acquired skill. \par Well, let's think about this for a minute. Suppose you want your character to have a high level of expertise in pistol. If he starts off with a high aptitude in hand-eye coordination, it stands to reason then that practising this particular skill will raise his ability level above that point. \par So, we've established that the level of expertise in a particular skill is based on the attribute that governs it -- in our example, pistol is based on Coordination. \par Characters usually begin with 7D in skill dice. Divide these dice among the skills the character possesses (defined by the character template or selected from the skill list provided by the gamemaster) just like attributes, except that the number of skill dice is added to the base attribute. For example, if the character had a Coordination of 2D+1 and you spent 1D od skill dice on blaster (a futuristic weapon), he would have a total blaster die code of 3D+1 (2D+1 + 1D=3D+1).
		\textit{\begin{itemize}
				\setlength\itemsep{-0.4em}
				\item[] \textbf{Example: Revised Space Smuggler Character -- 7 Total Skill Dice}
				\item[] Strength: 4D  \\ Resist damage 4d+2
				\item[] Reflexes: 3D+2 \\ Dodge 4D+2, starship piloting 5D
				\item[] Coordination: 2D+1 \\ Blaster 3D+1
				\item[] Perception: 4D \\ Con 5D+1, search 4D+2
				\item[] Reasoning: 2D
				\item[] Knowledge: 2D \\ Starports 3D
		\end{itemize}}
		Time to check the math. We spent 2 pips on the \textit{Resist damage} skill, 1D on \textit{dodge}, 1D+1 on \textit{starship piloting}, 1D on \textit{blaster}, 1D+1 on \textit{con}, 2 pips on \textit{search}, and 1D on \textit{starports}. Add up the dice (1D+1D+1D+1D+1D=5D) and the pips (2+1+1+2=6=2D) and we get a total of 7D (5D+2D=7D). \par Note that the standard limit on the number of skill dice you can add to any one skill for a starting character is 2D. Again, the gamemaster may change this requirement, but she'll tell you if that's the case. \par You can also change the number of skill dice by selecting Advantages and Disadvantages for your character.
		\subsection*{CHARACTER POINTS}
		Characters usually begin play with five Character Points (unless the gamemaster decides otherwise). The role Character Points play in the game will become apparent in the upcoming sections \textit{Making Dice Rolls} and \textit{Evolving Characters}.
		\begin{mytextbox}
			\subsection*{THE CHARACTER CREATION TEMPLATE}
				The gamemaster may provide you with a Character Creation Template that defines the various aspects of characters in her game world -- which attributes they have, which skills are available, how many attriute and skill dice they start with, allowed species, et cetera. From this 
		\end{mytextbox}	
	\end{multicols}
	\backmatter
	\section*{OGL OPEN GAME LICENSE Version 1.0a}
	{\footnotesize The following text is the property of Wizards of the Coast, Inc. and is Copyright 2000 Wizards of the Coast, Inc ("Wizards"). All Rights Reserved. \par
	\textbf{1. Definitions:} (a)"Contributors" means the copyright and/or trademark owners who have contributed Open Game Content; \\ (b)"Derivative Material" means copyrighted material including derivative works and translations (including into other computer languages), potation, modification, correction, addition, extension, upgrade, improvement, compilation, abridgment or other form in which an existing work may be recast, transformed or adapted; (c) "Distribute" means to reproduce, license, rent, lease, sell, broadcast, publicly display, transmit or otherwise distribute; (d)"Open Game Content" means the game mechanic and includes the methods, procedures, processes and routines to the extent such content does not embody the Product Identity and is an enhancement over the prior art and any additional content clearly identified as Open Game Content by the Contributor, and means any work covered by this License, including translations and derivative works under copyright law, but specifically excludes Product Identity. (e) "Product Identity" means product and product line names, logos and identifying marks including trade dress; artifacts; creatures characters; stories, storylines, plots, thematic elements, dialogue, incidents, language, artwork, symbols, designs, depictions, likenesses, formats, poses, concepts, themes and graphic, photographic and other visual or audio representations; names and descriptions of characters, spells, enchantments, personalities, teams, personas, likenesses and special abilities; places, locations, environments, creatures, equipment, magical or supernatural abilities or effects, logos, symbols, or graphic designs; and any other trademark or registered trademark clearly identified as Product identity by the owner of the Product Identity, and which specifically excludes the Open Game Content; (f) "Trademark" means the logos, names, mark, sign, motto, designs that are used by a Contributor to identify itself or its products or the associated products contributed to the Open Game License by the Contributor (g) "Use", "Used" or "Using" means to use, Distribute, copy, edit, format, modify, translate and otherwise create Derivative Material of Open Game Content. (h) "You" or "Your" means the licensee in terms of this agreement. \par
	\textbf{2. The License:} This License applies to any Open Game Content that contains a notice indicating that the Open Game Content may only be Used under and in terms of this License. You must affix such a notice to any Open Game Content that you Use. No terms may be added to or subtracted from this License except as described by the License itself. No other terms or conditions may be applied to any Open Game Content distributed using this License. \par
	\textbf{3. Offer and Acceptance:} By Using the Open Game Content You indicate Your acceptance of the terms of this License. \par
	\textbf{4. Grant and Consideration:} In consideration for agreeing to use this License, the Contributors grant You a perpetual, worldwide, royalty-free, non exclusive license with the exact terms of this License to Use, the Open Game Content. \par
	\textbf{5. Representation of Authority to Contribute:} If You are contributing original material as Open Game Content, You represent that Your Contributions are Your original creation and/or You have sufficient rights to grant the rights conveyed by this License. \par
	\textbf{6. Notice of License Copyright:} You must update the COPYRIGHT NOTICE portion of this License to include the exact text of the COPYRIGHT NOTICE of any Open Game Content You are copying, modifying or distributing, and You must add the title, the copyright date, and the copyright holder's name to the COPYRIGHT NOTICE of any original Open Game Content you Distribute. \par 
	\textbf{7. Use of Product Identity:} You agree not to Use any Product Identity, including as an indication as to compatibility, except as expressly licensed in another, independent Agreement with the owner of each element of that Product Identity. You agree not to indicate compatibility or co-adaptability with any Trademark or Registered Trademark in conjunction with a work containing Open Game Content except as expressly licensed in another, independent Agreement with the owner of such Trademark or Registered Trademark. The use of any Product Identity in Open Game Content does not constitute a challenge to the ownership of that Product Identity. The owner of any Product Identity used in Open Game Content shall retain all rights, title and interest in and to that Product Identity. \par 
	\textbf{8. Identification:} If you distribute Open Game Content You must clearly indicate which portions of the work that you are distributing are Open Game Content. \par 
	\textbf{9. Updating the License:} Wizards or its designated Agents may publish updated versions of this License. You may use any authorized version of this License to copy, modify and distribute any Open Game Content originally distributed under any version of this License. \par 
	\textbf{10. Copy of this License:} You MUST include a copy of this License with every copy of the Open Game Content You Distribute. \par 
	\textbf{11. Use of Contributor Credits:} You may not market or advertise the Open Game Content using the name of any Contributor unless You have written permission from the Contributor to do so. \par
	\textbf{12. Inability to Comply:} If it is impossible for You to comply with any of the terms of this License with respect to some or all of the Open Game Content due to statute, judicial order, or governmental regulation then You may not Use any Open Game Material so affected. \par 
	\textbf{13. Termination:} This License will terminate automatically if You fail to comply with all terms herein and fail to cure such breach within 30 days of becoming aware of the breach. All sublicenses shall survive the termination of this License. \par 
	\textbf{14. Reformation:} If any provision of this License is held to be unenforceable, such provision shall be reformed only to the extent necessary to make it enforceable. \par 
	\textbf{15. COPYRIGHT NOTICE} \\ Open Game License v 1.0 Copyright 2000,  Wizards of the Coast, Inc. \\ D6 Adventure (WEG51011), Copyright 2004, Purgatory Publishing Inc. \\ West End Games, WEG, and D6 System are trademarks and properties of Purgatory Publishing Inc. \par 
	\textbf{PRODUCT IDENTIFICATION:} \\ Product Identity: The D6 System; the D6 trademarks, the D6 and related logos and any derivative trademarks not specified as Open Game Content; and all cover and interior art and trade dress are designated as Product Identity (PI) and are properties of Purgatory Publishing Inc. All rights reserved. \\ Open Game Content: All game mechanics and material not covered under Product Identity (PI) above; OpenD6 trademark and OpenD6 logo (as displayed on this document cover page). \par}
\end{document}